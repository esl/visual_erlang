\documentclass[a4paper, pdftex]{tufte-handout}

%\usepackage{a4wide}
\usepackage{visual-erlang}
\usepackage{subfig}


\geometry{vmargin=2cm,width=180mm}

%\usepackage[T1]{fontenc}
%\usepackage{dejavu}


\title{Visual Erlang}
\author{Torben Hoffmann}%{e-mail: torben.lehoff at gmail.com}}
\date{Version 0.1.0}

\begin{document}
\maketitle


\begin{abstract}
  Visual Erlang is a notation for describing architectures of Erlang systems
  \sidenote{\textbf{NOTE:} this is work in progress, so there are things related to
    both the graphical and textual notation that has to be improved based on the
    feedback received.}.

  The notation is a dual graphical and textual version with a one to one mapping
  between the two.

  It is not the intention that all details of Erlang programs can be captured in
  Visual Erlang --- the focus is on making it easier to communicate the architecture.
\end{abstract}



The two building blocks of Erlang --- processes and modules -- will be used to attach
the functionality of a system.

Erlang uses message passing, so there will be a symbol for that in Visual Erlang, but
Visual Erlang follows the good practice that most things should be hidden behind an
API, e.g., if you have a \verb|gen_server| you will not send messages to it directly
using \verb|gen_server:call/2|, but rather provide an API function in a module that
will do it for you.

However, there are cases where message passing is the right way to solve a problem,
and then it is nice to have a notation for message passing.

The caption of the figures showing how a certain element of Visual Erlang looks like
will have the textual version of the same information. 

\section{Visual Erlang Symbols}
\label{sec:visu-erlang-symb}

\subsection{Processes}
\label{sec:processes}

Visual Erlang uses an ellipse to represent a process since a circle is often hard to
size nicely when the process names are long (Figure~\ref{fig:ve-process}). Luckily, a
circle is also an ellipse, so no harm is done if you use a circle.


\begin{marginfigure}
  \begin{center}
    \begin{tikzpicture}
      \node [ve-process] {my\_process};
    \end{tikzpicture}

    \caption{\texttt{Process my\_process.}}
    \label{fig:ve-process}
  \end{center}
\end{marginfigure}





If a process traps exits it should be drawn with a double lined ellipse
(Figure\ref{fig:ve-process-traps-exits}. 

\begin{marginfigure}
  \begin{center}
    \begin{tikzpicture}
      \node [ve-process-traps-exits] {P};
    \end{tikzpicture}
    \caption{\texttt{Process P traps exits.}}
    \label{fig:ve-process-traps-exits}
  \end{center}
\end{marginfigure}


Since dealing with failures is a very natural thing in Erlang we have notation for
monitoring of processes. When $P_1$ monitors $P_2$ it is drawn as a double line
ending with a circle (Figure~\ref{ve-monitoring}).

\begin{marginfigure}
  \centering
    \begin{tikzpicture}
      \node [ve-process] (p1) {$P_1$};
      \node [ve-process, right=of p1] (p2) {$P_2$};
      \draw [ve-monitors] (p1) -- (p2); 
    \end{tikzpicture}
  
  \caption{\texttt{P1 monitors P2.}}
  \label{fig:ve-monitoring}
\end{marginfigure}


Linking is the dual of monitoring with the extra twist that the linked processes will
die together. Linking is drawn with circles at both ends of a double line
(Figure~\ref{fig:ve-linking}). 


\begin{marginfigure}
  \centering
    \begin{tikzpicture}
      \node [ve-process] (p1) {$P_1$};
      \node [ve-process, right=of p1] (p2) {$P_2$};
      \draw [ve-link] (p1) -- (p2); 
    \end{tikzpicture}
  \caption{\texttt{P1 and P2 are linked.}}
  \label{fig:ve-linking}
\end{marginfigure}


\newpage

Especially for supervision trees it is important which process has spawned another
process, so in order to show that $P_1$ has spawned $P_2$ an arrow head is drawn
towards $P_2$ (Figure~\ref{fig:ve-spawn}).

\begin{marginfigure}
  \centering
    \begin{tikzpicture}
      \node [ve-process] (p1) {$P_1$};
      \node [ve-process, right=of p1] (p2) {$P_2$};
      \draw [ve-spawn] (p1) -- (p2); 
    \end{tikzpicture}
  \caption{\texttt{P1 spawns P2.}}
  \label{fig:ve-spawn}
\end{marginfigure}

For supervisors it is the case that they both spawns and links to their child
processes. This is shown by combining the symbols for spawning and linking
(Figure~\ref{fig:ve-spawn-linking}).

\begin{marginfigure}
  \centering
    \begin{tikzpicture}
      \node [ve-process] (p1) {$P_1$};
      \node [ve-process, right=of p1] (p2) {$P_2$};
      \draw [ve-spawn-link] (p1) -- (p2); 
    \end{tikzpicture}
  \caption{\texttt{P1 spawn-links P2.}}
  \label{fig:ve-spawn-linking}
\end{marginfigure}


\subsection{Messages and Trap Exit Signals}
\label{sec:messages-trap-exit}

Showing the sending of messages is meant to be a low-level thing and not to be the
guiding principle in Erlang systems. Most of the time message passing will hidden
inside modules like the components from the OTP library encourages. In the situations
where you have to send a message it is shown as a curly line going from sender to
receiver with the message on top of the line (Figure~\ref{fig:ve-send-message}).

\begin{marginfigure}
  \centering
    \begin{tikzpicture}
      \node [ve-process] (p1) {$P_1$};
      \node [ve-process, right=of p1] (p2) {$P_2$};
       \path (p1) edge [ve-send]  node [above, align=center, midway] {M} (p2);
    \end{tikzpicture}
  \caption{\texttt{P1 sends M to P2.}}
  \label{fig:ve-send-message}
\end{marginfigure}


Trap exit signals are turned into \texttt{'DOWN'} messages when a process is trapping
exits. The explicit flow of a trap exit message is shown with a zigzag line
(Figure~\ref{fig:ve-trap-exit}). 

\begin{marginfigure}
  \centering
    \begin{tikzpicture}
      \node [ve-process-traps-exits] (p1) {$P_1$};
      \node [ve-process, right=of p1] (p2) {$P_2$};
      \path (p1) edge [ve-link] (p2);
      \path (p2) edge [ve-exit-signal, bend left]  (p1);
    \end{tikzpicture}  
  \caption{Trap exit handling}
  \label{fig:ve-trap-exit}
\end{marginfigure}

This way of showing trap exits does not say much about what happens inside $P_1$ when
the \texttt{'DOWN'} message is received. When you want to say something about that
you can designate an abstract functionality to a process as shown in
Figure~\ref{fig:ve-abstract-functionality}. ``Why not write exactly which part of the
code handles the trap exit?'' you might ask. The reason is that in many cases you are
only interested in seeing what happens as a result of receiving a message, in
particular when it is a \texttt{'DOWN'} message; exactly which function in a module
that handles it is not of interest when communicating at the architecture level.

\begin{marginfigure}
  \centering
  \begin{tikzpicture}
    \node [ve-process-traps-exits] (p1) {$P_1$};
    \node [ve-process, right=of p1] (p2) {$P_2$};
    \path (p1) edge [ve-link] (p2);
    \node [ve-functionality, below=of p1] (p1-f1) {handle\_down};
    \path (p1) edge (p1-f1);
    \path (p2) edge [ve-exit-signal] node [below right, align=center, midway] {'DOWN'} (p1-f1);
  \end{tikzpicture}
  \caption{\texttt{Process P1 has functionality handle\_down.}\\
    \texttt{Functionality handle\_down handles 'DOWN' message.}}
  \label{fig:ve-abstract-functionality}
\end{marginfigure}


\subsection{Modules, Processes and APIs}
\label{sec:modul-proc-apis}

A corner stone in good Erlang programming --- or most programming for that matter ---
is to provide APIs that encapsulates the implementation.

APIs in Erlang are given as exported functions from a module.

A module exporting a function is shown as a double rectangle for the module and a
diamond arrow head from the module to the function it exports. Functions are shown as
rectangles containing the name and arity of the function. See
Figure~\ref{fig:ve-module-api}.

\begin{marginfigure}
  \centering
  \begin{tikzpicture}
    \node [ve-module] (m1) {\verb|my_mod|};
    \node [ve-function, below=of m1] (f) {\verb|f/N|};
    \path (m1) edge [ve-has-api] (f);
  \end{tikzpicture}
  \caption{\texttt{Module my\_mod has API f/N}}
  \label{fig:ve-module-api}
\end{marginfigure}

Processes are --- in general --- started through a module. This is shown as a filled
diamond arrow head pointing towards the module as in Figure~\ref{}.

\begin{marginfigure}
  \centering
  \begin{tikzpicture}
    \node [ve-module] (m1) {\verb|my_mod|};
    \node [ve-process, below=of m1] (p1) {$P_1$};
    \path (m1) edge [ve-has-instance] (p1);
  \end{tikzpicture}
  \caption{\texttt{Module my\_mod has instance P1.}}
  \label{fig:ve-module-process-instance}
\end{marginfigure}

For the majority of the cases the process will be what is referred to and in that
case one can use the shorthand in
Figure~\ref{fig:ve-module-process-instance-shorthand}, where the name of the module
that $P_1$ is an instance of is given after the process name separated with a colon. 

\begin{marginfigure}
  \centering
  \begin{tikzpicture}
    \node [ve-process] (p1) {$P_1$:\verb|my_mod|};
  \end{tikzpicture}
  \caption{\texttt{Process P1 is an instance of  my\_mod.}}
  \label{fig:ve-module-process-instance-shorthand}
\end{marginfigure}

\newpage

Since the name of the module is often used to refer to the process when there is only
one instance of process you can give the name of the module surrounded by colons as a
shorthand as in Figure~\ref{fig:ve-module-process-instance-shorthand2}.

\begin{marginfigure}
  \centering
  \begin{tikzpicture}
    \node [ve-process] (p1) {\verb|:my_mod:|};
  \end{tikzpicture}
  \caption{\texttt{ModuleProcess my\_mod.}}
  \label{fig:ve-module-process-instance-shorthand2}
\end{marginfigure}

When an API in a module towards a process always involves interaction with the
process it is permitted to associate functions of the module with the process
instance as shown in Figure~\ref{fig:ve-process-api}.

\begin{marginfigure}
  \centering
  \begin{tikzpicture}
    \node [ve-process] (p1) {$P_1$:\verb|my_mod|};
    \node [ve-function, below=of m1] (f) {\verb|f/N|};
    \path (p1) edge [ve-has-api] (f);
  \end{tikzpicture}
  \caption{\texttt{Process P1 has API f/N.}}
  \label{fig:ve-process-api}
\end{marginfigure}

This is the predominant situation, but in some cases one has functions that do not
involve the process and then you see diagrams like in
Figure~\ref{fig:ve-module-process-api-sharing}. This allows you to clearly
communicate which functions of the API that involves the process and which are
``merely'' regular functions.

\begin{marginfigure}
  \centering
  \begin{tikzpicture}
    \node [ve-module] (m) {\verb|my_mod|};
    \node [ve-function, below=of m] (f) {\verb|f/N|};
    \path (m) edge[ve-has-api] (f);
    \node [ve-process, below right=of m] (p) {$P_1$};
    \path (m) edge [ve-has-instance] (p);
    \node [ve-function, below=of p] (g) {\verb|g/N|};
    \path (p) edge[ve-has-api] (g);
  \end{tikzpicture}
  \caption{\texttt{Module my\_mod has instance P1.}\\
    \texttt{my\_mod has API f/N}\\
    \texttt{P1 has API g/N}
  }
  \label{fig:ve-module-process-api-sharing}
\end{marginfigure}

\subsection{Invoking functions}
\label{sec:invoking-functions}

Since APIs are given in terms of functions we need a way to invoke the functions and
that is done using an arrow ending with a triangle arrow head from the caller to the
function as seen in Figure~\ref{}. Above the arrow you can see the arguments
(either the types or some saying names) and below it you can see a short arrow
followed by the result to be returned to the caller. If the result isn't important it
can be left out.

\begin{figure}[h]
  \centering
  \begin{tikzpicture}
    \node [ve-module] (m1) {\verb|m1|};
    \node [ve-function, below=of m] (f) {\verb|f/N|};
    \path (m) edge[ve-has-api] (f);
    \node [ve-module, right=5cm of m1] (m2) {\verb|m2|};
    \node [ve-function, below=of m2] (g) {\verb|g/N|};
    \path (m2) edge[ve-has-api] (g);
    \vefunctioncallreturn{f}{g}{(arg1, arg2)}{\vereturnleft{sometype()}}
  \end{tikzpicture}
  \caption{tbd}
  \label{fig:ve-invoke-function}
\end{figure}

\subsection{Missing Notation}
\label{sec:missing-notation}

\begin{itemize}
\item ETS tables.
\item states in \verb|gen_fsm|   
\end{itemize}



\end{document}


%%% Local Variables:
%%% mode: latex
%%% TeX-master: t
%%% End:
